Adaptive Group Normalization (AGN) introduces a dynamic approach to neural network normalization, clustering channels based on statistical properties through an adapted clustering algorithm. This method surpasses Group Normalization (GN) by enabling variable group sizes and aligning channels with similar statistics for more effective normalization. AGN enhances model adaptability and accuracy by providing flexible normalization that can be tailored to specific characteristics of the data, leading to improved learning dynamics. Furthermore, AGN's ability to dynamically adjust to the changing distribution of features within a network promotes more stable and faster convergence, offering significant advancements in deep learning and computer vision. This adaptability is particularly advantageous in scenarios where GN's fixed group sizes limit performance, making AGN a promising alternative for a wide range of applications. Our experiments on classification, object detection, and segmentation tasks across benchmark datasets and architectures underscore AGN's superiority over GN.